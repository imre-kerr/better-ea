\documentclass[a4paper,12pt]{article}
\usepackage{graphicx}
\usepackage[cm]{fullpage}
\author{Didrik Jonassen, Imre Kerr\vspace{-2ex}}
\title{\vspace{-5ex}Project 3\\ IT3708 --- Subsymbolic methods in AI}
\date{\today}

\begin{document}

\maketitle

\section{Assumptions}
\paragraph{}Since this wasn't entirely clear in the problem text, we worked under the assumption that the position of the tracker and the internal state of the agent should be reset between each object drop.

\section{Description of Code}
\subsection{Genotype Representation}
\paragraph{}Once again we opted to go for the float-list genotype representation. The genotype consists of the following floats, each with the ranges given in the problem text:
\begin{itemize}
\item{22 weights}
\item{4 biases}
\item{4 gains}
\item{4 time constants}
\end{itemize}

\subsection{CTRNN}
\paragraph{}The CTRNN is modeled as an object with a method \texttt{timestep}. This method accepts a list of sensor inputs, and returns a list of neuron outputs. It contains lists (one for each layer) of node (neuron) objects. These nodes each have list of parent nodes, a \texttt{timestep} method that computes the new output level, and an \texttt{update} method that actually updates the output level. This is necessary to make sure each node sees the same output level from a given node. Additionally the CTRNN has a \texttt{reset} method that resets the outputs and internal state.

\subsection{The Game}
\paragraph{}Each generation, a single \texttt{Game} object is instantiated. This guarantees that the positions and sizes of the objects are equal for each agent tested. The object has a single method \texttt{play}, which takes a CTRNN object and a boolean value for visualization or no visualization, and returns the score from that gameplay round. In the general case we give 1 point for catching a small object, and 1.2 points for avoiding a large object. Failing to catch/avoid gives no points. 1.2 points for avoidance may seem strange, but it did give better results than 1. This suggests that a uniform size distribution may not optimally incentivize avoidance.

\section{Catching Behavior}
\subsection{Methodology}
\paragraph{}For evolving only catching behavior, we modified the game class to give 0 points both for catching and avoiding large objects. Small objects still had to be caught entirely to give points. The fitness function plays a game with each agent, and then sets the fitness to $\frac{prev\_fitness + score}{2}$. This exponential moving average method helps us avoid ``lucky idiots'' by testing each agent multiple times.
\paragraph{}The EA parameters used were as follows:\\
\begin{tabular}{ll}
\hline
Parameter & Value \\
\hline \hline
Population size & 50 \\
Adult selection & Generational mixing \\
Litter size & 50 \\
Parent selection & Sigma scaling \\
Mutation type & Gaussian \\
Mutation rate & 0.06 \\
Mutation std.dev. & 0.2 \\
Crossover type & Random choice \\
Max generations & 100 \\
\hline
\end{tabular}

\subsection{Results}
\paragraph{}Over 20 runs, 17 of the agents caught every small object. In one case, one was missed by running past it, and in two others there was a partial catch.

\section{Catching and Avoidance}
\subsection{Methodology}
\paragraph{}For evolving catching and avoidance, we used the same exponential moving average method as for catching only, but of course we used the standard scoring (1 point for a correct catch, 1.2 points for a correct avoidance).
\paragraph{}Since catching and avoidance are harder than just catching, we upped the population size and max generations:\\
\begin{tabular}{ll}
\hline
Parameter & Value \\
\hline \hline
Population size & 100 \\
Adult selection & Generational mixing \\
Litter size & 100 \\
Parent selection & Sigma scaling \\
Mutation type & Gaussian \\
Mutation rate & 0.06 \\
Mutation std.dev. & 0.2 \\
Crossover type & Random choice \\
Max generations & 300 \\
\hline
\end{tabular}

\subsection{Results}
\paragraph{}We ran the test three times. The first time, we got an agent that tried to catch everything. The second time, we got one with an interesting, but inconsistent, avoidance strategy. The last one worked almost perfectly, and an expanded description follows here:
\paragraph{}To begin with, the tracker moves right at a speed of two units per turn. It keeps doing this until it sees an object, at which point it slows down to determine the size. If it's a small object, it stops and catches it. If it's a large object, it speeds up and then stops to the right of the object. It will never move left, but two units per turn gives plenty of time to wrap around and reach an object directly to the left, and get out of harm's way if necessary.

\end{document}
